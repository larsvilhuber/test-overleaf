
From the results above, it is clear that current commuting zone definitions understate the uncertainty of zone assignment, which has implications for empirical results. Importantly, this uncertainty manifests itself on two different margins: uncertainty about zone assignment due to errors in the flows, and uncertainty about zone assignment due to the chosen cutoff. 

Given this uncertainty, we have two pieces of advice for researchers using commuting zones. First, we suggest displaying results for a variety of different cluster counts resulting from a range of cutoff values. This point is particularly important for researchers applying the methodology from TS to new datasets or for characterizing labor markets outside the United States, given that cluster counts are subjective and that results can differ considerably based on the count. Second, we suggest re-estimating results using multiple realizations of commuting zones, which incorporates the additional uncertainty because of the underlying error in the measurement of flows. Researchers can validate results by examining either the distribution of $\beta$ or the distribution of the t-statistics, as described in the previous sub-sections. 

To aid researchers in this effort, we provide datasets and code online that include a crosswalk from county to all the realizations of commuting zones used in this paper to characterize uncertainty in inputs as as well as different cutoff values.\footnote{Our code is available at \iftoggle{blind}{[URL suppressed]}{\url{https://github.com/larsvilhuber/MobZ/}}, see also \iftoggle{blind}{[self-citation suppressed]}{\citet{mobzrepl201704}}.} We also provide our sample code that produced Figures \ref{fig:1990dist} and \ref{fig:cutoff_dist}.

Numerous influential papers in labor economics have used commuting zones as an alternative definition to local labor markets. However, researchers typically do not evaluate how the methodology used to construct commuting zones may impact their findings, nor have there been any evaluations of the sensitivity of commuting zones to design feature more generally. Our paper contributes to this literature by analyzing this methodology and its implications for empirical applications.

We document that the commuting zone methodology is sensitive to uncertainty in the input data and parameter choices and we demonstrate how these features affect the resulting labor market definitions. Furthermore, we demonstrate that uncertainty in local labor market definitions also affects empirical estimates that use commuting zones as a unit of analysis. 

Future work may explore other clustering methods, which are less history-dependent, as they may be better suited for considering a wide range of cluster counts and for evaluating the optimality of cluster counts. Developing metrics to compare zone configurations against one another will facilitate comparisons of the overlap of different clustering outcomes. A more complete characterization of measurement error in the flow measures, reflecting the sparse nature of survey responses, may improve the economic interpretation of flows in rural areas and for long distance flows. Additional metrics of local labor market integration may help to evaluate the overall validity of various definitions.

